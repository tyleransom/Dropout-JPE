% source is T14truncstata.tex
\begin{table}[ht]
\caption{College completion status frequencies: data, baseline model, static model}
\label{tab:CCDOSOfit}
\centering{}
\begin{threeparttable}
\begin{tabular}{lccc}
\toprule
                                          &      & Baseline & Static \\
Status                                    & Data & model    & model  \\
\midrule
Continuous completion (CC), Science&2.88&3.11&3.13 \\
Continuous completion (CC), Non-Science&7.21&7.97&7.75 \\
Stop out (SO) but graduated Science&0.24&0.50&0.49 \\
Stop out (SO) but graduated Non-Science&1.23&1.74&1.66 \\
Stop out (SO) then drop out&2.70&5.72&5.92 \\
Drop out (DO)&20.83&25.32&24.79 \\
Never went to college&28.61&33.13&34.44 \\
Truncated&36.30&22.52&21.83 \\
\bottomrule
\end{tabular}
\footnotesize Notes: Model frequencies are constructed using 10 simulations of the structural model for each individual included in the estimation. Counterfactual frequencies use 10 simulations of each counterfactual model. We set the panel length in the model to be the same as the panel length in the data. This is because the model assumes random attrition conditional on all observables and unobservables. 

\medskip

Completion status is computed on the first 10 periods of data (i.e. assuming that college is not an option after period 10).

\medskip

``Truncated'' refers to those who were enrolled in period 10.
\end{threeparttable}
\end{table}
