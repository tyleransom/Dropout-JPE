\begin{table} 
\caption{Wage Decompositions} 
\label{tab:wageDecomp} 
\centering{} 
\subfloat[Average full-time log wage and choice share by employment sector and education level at age 28 in baseline and counterfactual models]{ 
\resizebox{\textwidth}{!}{ 
\begin{threeparttable} 
\begin{tabular}{lcccccc} 
\toprule 
 & \multicolumn{3}{c}{Average full-time log wage, relative to} & & & \\ 
 & \multicolumn{3}{c}{blue-collar non-graduates in baseline} & \multicolumn{3}{c}{Choice shares (\%)} \\ 
\cmidrule(r){2-4}\cmidrule(l){5-7} 
Sector and Education Level & Baseline & Counterfactual & No Frictions Cfl & Baseline & Counterfactual & No Frictions Cfl \\ 
\midrule 
White collar, Science graduate & 0.458 & 0.904 & 0.868 & 3.435 & 5.535 & 6.865 \\ 
White collar, Non-Science graduate & 0.317 & 0.632 & 0.557 & 8.148 & 5.278 & 7.248 \\ 
White collar, Non-graduate & 0.110 & 0.345 & 0.146 & 5.457 & 0.961 & 4.717 \\ 
Blue collar, Science graduate & 0.197 & 0.298 & 0.301 & 2.452 & 5.383 & 4.865 \\ 
Blue collar, Non-Science graduate & 0.111 & 0.082 & 0.078 & 6.617 & 9.617 & 8.461 \\ 
Blue collar, Non-graduate & 0.000 & 0.009 & 0.007 & 41.400 & 49.570 & 43.804 \\ 
Remainder & --- & --- & ---  & 32.491 & 23.657 & 24.039 \\ 
\bottomrule 
\end{tabular} 
\footnotesize Notes: ``No Frictions Cfl'' refers to the counterfactual where white-collar work is always an option. Columns in the ``choice shares'' panel sum to 100. 
\end{threeparttable} 
} 
} 
\bigskip 
\bigskip 
\subfloat[Full-time log wage premia at age 28 in baseline and counterfactual models]{ 
\resizebox{\textwidth}{!}{ 
\begin{threeparttable} 
\begin{tabular}{lcccccc} 
\toprule 
& & & & \multicolumn{3}{c}{Change in premium (relative to baseline)} \\
 & \multicolumn{3}{c}{Full-time log wage premium} & \multicolumn{3}{c}{due to better sorting on abilities} \\ 
\cmidrule(r){2-4}\cmidrule(l){5-7} 
Sector & Baseline & Counterfactual & No Frictions Cfl & Baseline & Counterfactual & No Frictions Cfl \\ 
\midrule 
College wage premium & 0.247 & 0.400 & 0.421 & --- & 0.137 & 0.140 \\ 
Science college premium & 0.336 & 0.590 & 0.612 & --- & 0.213 & 0.217 \\ 
Non-science college premium & 0.212 & 0.262 & 0.279 & --- & 0.075 & 0.077 \\ 
White-collar wage premium & 0.255 & 0.692 & 0.525 & --- & 0.321 & 0.229 \\ 
\bottomrule 
\end{tabular} 
\footnotesize Notes: ``College wage premium'' is the difference in average log wages between college graduates (regardless of major) and non-graduates. ``Science college premium''  is the difference in average log wages between science graduates and non-graduates. ``Non-science college premium''  is the difference in average log wages between non-science graduates and non-graduates. ``White collar premium''  is the difference in average log wages between white-collar and blue-collar workers. 
 
\medskip{} 
 
For the panel on changes in premia, numbers represent differences in differences in average abilities (in log dollar units). The first difference is between sector groups (e.g. college graduates vs. non-graduates) and the second difference is between counterfactual and baseline. We compress the bivariate work ability distribution into a single ability index based on which sector each full-time worker is working in. 
\end{threeparttable} 
} 
} 
\end{table} 
